%% Generated by Sphinx.
\def\sphinxdocclass{report}
\documentclass[letterpaper,10pt,english]{sphinxmanual}
\ifdefined\pdfpxdimen
   \let\sphinxpxdimen\pdfpxdimen\else\newdimen\sphinxpxdimen
\fi \sphinxpxdimen=.75bp\relax

\PassOptionsToPackage{warn}{textcomp}
\usepackage[utf8]{inputenc}
\ifdefined\DeclareUnicodeCharacter
% support both utf8 and utf8x syntaxes
\edef\sphinxdqmaybe{\ifdefined\DeclareUnicodeCharacterAsOptional\string"\fi}
  \DeclareUnicodeCharacter{\sphinxdqmaybe00A0}{\nobreakspace}
  \DeclareUnicodeCharacter{\sphinxdqmaybe2500}{\sphinxunichar{2500}}
  \DeclareUnicodeCharacter{\sphinxdqmaybe2502}{\sphinxunichar{2502}}
  \DeclareUnicodeCharacter{\sphinxdqmaybe2514}{\sphinxunichar{2514}}
  \DeclareUnicodeCharacter{\sphinxdqmaybe251C}{\sphinxunichar{251C}}
  \DeclareUnicodeCharacter{\sphinxdqmaybe2572}{\textbackslash}
\fi
\usepackage{cmap}
\usepackage[T1]{fontenc}
\usepackage{amsmath,amssymb,amstext}
\usepackage{babel}
\usepackage{times}
\usepackage[Bjarne]{fncychap}
\usepackage{sphinx}

\fvset{fontsize=\small}
\usepackage{geometry}

% Include hyperref last.
\usepackage{hyperref}
% Fix anchor placement for figures with captions.
\usepackage{hypcap}% it must be loaded after hyperref.
% Set up styles of URL: it should be placed after hyperref.
\urlstyle{same}
\addto\captionsenglish{\renewcommand{\contentsname}{Contents:}}

\addto\captionsenglish{\renewcommand{\figurename}{Fig.}}
\addto\captionsenglish{\renewcommand{\tablename}{Table}}
\addto\captionsenglish{\renewcommand{\literalblockname}{Listing}}

\addto\captionsenglish{\renewcommand{\literalblockcontinuedname}{continued from previous page}}
\addto\captionsenglish{\renewcommand{\literalblockcontinuesname}{continues on next page}}
\addto\captionsenglish{\renewcommand{\sphinxnonalphabeticalgroupname}{Non-alphabetical}}
\addto\captionsenglish{\renewcommand{\sphinxsymbolsname}{Symbols}}
\addto\captionsenglish{\renewcommand{\sphinxnumbersname}{Numbers}}

\addto\extrasenglish{\def\pageautorefname{page}}

\setcounter{tocdepth}{1}



\title{Wordhack Documentation}
\date{Dec 01, 2019}
\release{0.0.1}
\author{Group 17 \\ Wells, Derick}
\newcommand{\sphinxlogo}{\vbox{}}
\renewcommand{\releasename}{Release}
\makeindex
\begin{document}

\pagestyle{empty}
\maketitle
\pagestyle{plain}
\sphinxtableofcontents
\pagestyle{normal}
\phantomsection\label{\detokenize{index::doc}}



\chapter{Installation}
\label{\detokenize{index:installation}}

Install requirements needed:

\fvset{hllines={, ,}}%
\begin{sphinxVerbatim}[commandchars=\\\{\}]
pip install \PYGZhy{}r requirements.txt
\end{sphinxVerbatim}

To start the game, run the game through Python 3 in the command terminal
with the following command (while in root directory):

On Windows:

\fvset{hllines={, ,}}%
\begin{sphinxVerbatim}[commandchars=\\\{\}]
py \PYGZhy{}3 main.py
\end{sphinxVerbatim}

On Linux:

\fvset{hllines={, ,}}%
\begin{sphinxVerbatim}[commandchars=\\\{\}]
python3 main.py
\end{sphinxVerbatim}


\chapter{Source Code Documentation}
\label{\detokenize{index:source-code-documentation}}

\section{modules/main.py}
\label{\detokenize{index:modules-main-py}}

\begin{fulllineitems}
\pysiglinewithargsret{\sphinxcode{\sphinxupquote{main.}}\sphinxbfcode{\sphinxupquote{main}}}{}{}
Starts the program

\end{fulllineitems}



\section{modules/interface.py}
\label{\detokenize{index:modules-interface-py}}

\begin{fulllineitems}
\pysigline{\sphinxbfcode{\sphinxupquote{class }}\sphinxcode{\sphinxupquote{interface.}}\sphinxbfcode{\sphinxupquote{Interface}}}
Handles all of GUI work as well as updating leaderboard and some game logic.
Utilizes one instance of Engine class throughout the whole program to prevent
dictionary trie loading time.

Refer to main.py for running program.
\begin{quote}\begin{description}
\item[{Variables}] \leavevmode\begin{itemize}
\item {} 
\sphinxstyleliteralstrong{\sphinxupquote{active\_buttons}} (\sphinxstyleliteralemphasis{\sphinxupquote{list}}) \textendash{} List of buttons in a certain screen.

\item {} 
\sphinxstyleliteralstrong{\sphinxupquote{lb\_list}} (\sphinxstyleliteralemphasis{\sphinxupquote{list}}) \textendash{} Leaderboard read from leaderboard.txt file (stored in ‘resources/data/’).

\item {} 
\sphinxstyleliteralstrong{\sphinxupquote{start\_is\_running}} (\sphinxstyleliteralemphasis{\sphinxupquote{bool}}) \textendash{} An indicator whether start menu is running.

\item {} 
\sphinxstyleliteralstrong{\sphinxupquote{diff\_is\_running}} (\sphinxstyleliteralemphasis{\sphinxupquote{bool}}) \textendash{} An indicator whether difficulty menu is running.

\item {} 
\sphinxstyleliteralstrong{\sphinxupquote{game\_is\_running}} (\sphinxstyleliteralemphasis{\sphinxupquote{bool}}) \textendash{} An indicator whether main game is running.

\item {} 
\sphinxstyleliteralstrong{\sphinxupquote{lb\_is\_running}} (\sphinxstyleliteralemphasis{\sphinxupquote{bool}}) \textendash{} An indicator whether leaderboard is running.

\item {} 
\sphinxstyleliteralstrong{\sphinxupquote{end\_is\_running}} (\sphinxstyleliteralemphasis{\sphinxupquote{bool}}) \textendash{} An indicator whether game end (game over) is running.

\item {} 
\sphinxstyleliteralstrong{\sphinxupquote{ascii}} (\sphinxstyleliteralemphasis{\sphinxupquote{Ascii}}) \textendash{} An instance of Ascii.

\item {} 
\sphinxstyleliteralstrong{\sphinxupquote{engine}} (\sphinxstyleliteralemphasis{\sphinxupquote{Engine}}) \textendash{} An instance of Engine to be active throughout program instance.

\item {} 
\sphinxstyleliteralstrong{\sphinxupquote{scp\_regular}} (\sphinxstyleliteralemphasis{\sphinxupquote{pyglet.font}}) \textendash{} A font container for labels.

\item {} 
\sphinxstyleliteralstrong{\sphinxupquote{scp\_bold}} (\sphinxstyleliteralemphasis{\sphinxupquote{pyglet.font}}) \textendash{} A font container for labels.

\item {} 
\sphinxstyleliteralstrong{\sphinxupquote{lb\_layout}} (\sphinxstyleliteralemphasis{\sphinxupquote{pyglet.image}}) \textendash{} Image loaded in leaderboard screen.

\item {} 
\sphinxstyleliteralstrong{\sphinxupquote{game\_layout}} (\sphinxstyleliteralemphasis{\sphinxupquote{pyglet.image}}) \textendash{} Image loaded in main game screen.

\item {} 
\sphinxstyleliteralstrong{\sphinxupquote{grid\_4x4}} (\sphinxstyleliteralemphasis{\sphinxupquote{pyglet.image}}) \textendash{} Part of 4x4 main game screen.

\item {} 
\sphinxstyleliteralstrong{\sphinxupquote{grid\_5x5}} (\sphinxstyleliteralemphasis{\sphinxupquote{pyglet.image}}) \textendash{} Part of 5x5 main game screen.

\item {} 
\sphinxstyleliteralstrong{\sphinxupquote{show\_idx}} (\sphinxstyleliteralemphasis{\sphinxupquote{int}}) \textendash{} Tracks leaderboard index in leaderboard screen.

\item {} 
\sphinxstyleliteralstrong{\sphinxupquote{curr\_button}} (\sphinxstyleliteralemphasis{\sphinxupquote{int}}) \textendash{} Tracks current selected button id.

\item {} 
\sphinxstyleliteralstrong{\sphinxupquote{select\_board\_size}} (\sphinxstyleliteralemphasis{\sphinxupquote{int}}) \textendash{} Represents user choice of 4x4 or 5x5.

\item {} 
\sphinxstyleliteralstrong{\sphinxupquote{game\_user\_input}} (\sphinxstyleliteralemphasis{\sphinxupquote{string}}) \textendash{} Contains user input for main game screen and
end game screen.

\item {} 
\sphinxstyleliteralstrong{\sphinxupquote{start}} (\sphinxstyleliteralemphasis{\sphinxupquote{float}}) \textendash{} Time game started. For game timer.

\item {} 
\sphinxstyleliteralstrong{\sphinxupquote{max\_time}} (\sphinxstyleliteralemphasis{\sphinxupquote{float}}) \textendash{} Maximum amount of time (in seconds) game instance will last.

\item {} 
\sphinxstyleliteralstrong{\sphinxupquote{answer\_correct}} (\sphinxstyleliteralemphasis{\sphinxupquote{string}}) \textendash{} For calling ASCII indicator label. Set to ‘n’
to clear indicator label.

\end{itemize}

\end{description}\end{quote}


\begin{fulllineitems}
\pysiglinewithargsret{\sphinxbfcode{\sphinxupquote{difficulty}}}{}{}
Sets difficulty-choice menu state

\end{fulllineitems}



\begin{fulllineitems}
\pysiglinewithargsret{\sphinxbfcode{\sphinxupquote{difficulty\_screen}}}{}{}
Loads difficulty menu assets onto window when called from update

\end{fulllineitems}



\begin{fulllineitems}
\pysiglinewithargsret{\sphinxbfcode{\sphinxupquote{end\_screen}}}{}{}
Loads game end menu (game over) assets onto window when called from update

\end{fulllineitems}



\begin{fulllineitems}
\pysiglinewithargsret{\sphinxbfcode{\sphinxupquote{game}}}{}{}
Sets main game state

\end{fulllineitems}



\begin{fulllineitems}
\pysiglinewithargsret{\sphinxbfcode{\sphinxupquote{game\_end}}}{}{}
Sets game end (game over) menu state

\end{fulllineitems}



\begin{fulllineitems}
\pysiglinewithargsret{\sphinxbfcode{\sphinxupquote{game\_screen}}}{}{}
Loads main game assets onto window when called from update

\end{fulllineitems}



\begin{fulllineitems}
\pysiglinewithargsret{\sphinxbfcode{\sphinxupquote{indicator}}}{}{}
Loads indicator label (“success”/”error”) onto
window during main game when answer is entered.

\end{fulllineitems}



\begin{fulllineitems}
\pysiglinewithargsret{\sphinxbfcode{\sphinxupquote{leaderboard}}}{}{}
Sets leaderboard menu state

\end{fulllineitems}



\begin{fulllineitems}
\pysiglinewithargsret{\sphinxbfcode{\sphinxupquote{leaderboard\_screen}}}{}{}
Loads leaderboard menu assets onto window when called from update

\end{fulllineitems}



\begin{fulllineitems}
\pysiglinewithargsret{\sphinxbfcode{\sphinxupquote{set\_to\_n}}}{\emph{dt}}{}
Helper function for indicator() method animation.
\begin{quote}\begin{description}
\item[{Parameters}] \leavevmode
\sphinxstyleliteralstrong{\sphinxupquote{dt}} (\sphinxstyleliteralemphasis{\sphinxupquote{int}}) \textendash{} Used by pyglet.clock.schedule(). Indicates delay in seconds.

\end{description}\end{quote}

\end{fulllineitems}



\begin{fulllineitems}
\pysiglinewithargsret{\sphinxbfcode{\sphinxupquote{start\_menu}}}{}{}
Sets start menu state

\end{fulllineitems}



\begin{fulllineitems}
\pysiglinewithargsret{\sphinxbfcode{\sphinxupquote{start\_screen}}}{}{}
Loads start menu assets onto window when called from update

\end{fulllineitems}



\begin{fulllineitems}
\pysiglinewithargsret{\sphinxbfcode{\sphinxupquote{stats}}}{}{}
Loads current game statistics as labels  and decorative
ASCII art from Ascii class onto window during main game.

\end{fulllineitems}



\begin{fulllineitems}
\pysiglinewithargsret{\sphinxbfcode{\sphinxupquote{timer}}}{\emph{current\_time}}{}
Loads timer label onto window during the main game
\begin{quote}\begin{description}
\item[{Parameters}] \leavevmode
\sphinxstyleliteralstrong{\sphinxupquote{current\_time}} (\sphinxstyleliteralemphasis{\sphinxupquote{int}}) \textendash{} From time.time() called in game\_screen() method.
Time (in seconds)

\end{description}\end{quote}

\end{fulllineitems}



\begin{fulllineitems}
\pysiglinewithargsret{\sphinxbfcode{\sphinxupquote{update}}}{\emph{dt}}{}
Contains main event loops

\end{fulllineitems}



\begin{fulllineitems}
\pysiglinewithargsret{\sphinxbfcode{\sphinxupquote{view\_as\_qu}}}{\emph{word}}{}
Helper function to convert ‘\$’ char in string to
user-friendly ‘qu’
\begin{quote}\begin{description}
\item[{Parameters}] \leavevmode
\sphinxstyleliteralstrong{\sphinxupquote{word}} (\sphinxstyleliteralemphasis{\sphinxupquote{string}}) \textendash{} Word to convert.

\end{description}\end{quote}

\end{fulllineitems}


\end{fulllineitems}



\section{modules/engine.py}
\label{\detokenize{index:modules-engine-py}}

\begin{fulllineitems}
\pysigline{\sphinxbfcode{\sphinxupquote{class }}\sphinxcode{\sphinxupquote{engine.}}\sphinxbfcode{\sphinxupquote{Engine}}}
This class handles the game logic.

Generates a random grid of letters and, aided by the Trie object, calculates all
possible words as well as storing other game-related information such as the maximum possible words
and the current score of the player. It is called by the Interface object on game runtime.
Throughout the whole instance of the game, only one Engine instance will be used, and is
therefore should not be called upon directly by the main method. Code for Engine class
and Trie were modified and are sourced from: \sphinxurl{https://github.com/macallmcqueen/boggle-solver-using-trie.git}.
\begin{quote}\begin{description}
\item[{Variables}] \leavevmode\begin{itemize}
\item {} 
\sphinxstyleliteralstrong{\sphinxupquote{trie}} (\sphinxstyleliteralemphasis{\sphinxupquote{Trie}}) \textendash{} The word trie formed by the current board configuartions.

\item {} 
\sphinxstyleliteralstrong{\sphinxupquote{max\_score}} (\sphinxstyleliteralemphasis{\sphinxupquote{int}}) \textendash{} The maximum possible score given possible words.

\item {} 
\sphinxstyleliteralstrong{\sphinxupquote{curr\_score}} (\sphinxstyleliteralemphasis{\sphinxupquote{int}}) \textendash{} The current score of game state.

\item {} 
\sphinxstyleliteralstrong{\sphinxupquote{game\_board\_size}} (\sphinxstyleliteralemphasis{\sphinxupquote{int}}) \textendash{} The dimensions of the current board. Has two possible values: 4 and 5.

\item {} 
\sphinxstyleliteralstrong{\sphinxupquote{game\_board}} (\sphinxstyleliteralemphasis{\sphinxupquote{list}}) \textendash{} A two-dimensional array of characters representing the current game board.

\item {} 
\sphinxstyleliteralstrong{\sphinxupquote{game\_solutions}} (\sphinxstyleliteralemphasis{\sphinxupquote{set}}) \textendash{} The set of all calculated possible words given a board and the dictionary of words.

\item {} 
\sphinxstyleliteralstrong{\sphinxupquote{game\_answered}} (\sphinxstyleliteralemphasis{\sphinxupquote{list}}) \textendash{} The list of correct answered so far in a game state.

\end{itemize}

\end{description}\end{quote}


\begin{fulllineitems}
\pysiglinewithargsret{\sphinxbfcode{\sphinxupquote{make\_board}}}{\emph{size}}{}
Creates NxN letter array and sets it to game\_board.
\begin{quote}\begin{description}
\item[{Parameters}] \leavevmode
\sphinxstyleliteralstrong{\sphinxupquote{size}} (\sphinxstyleliteralemphasis{\sphinxupquote{int}}) \textendash{} Represents N in an NxN board. Picked by user in Interface class.

\item[{Returns}] \leavevmode
no value

\end{description}\end{quote}

\end{fulllineitems}



\begin{fulllineitems}
\pysiglinewithargsret{\sphinxbfcode{\sphinxupquote{make\_lower\_case}}}{\emph{mat}}{}
Converts to lower case to avoid case-sensitivity

\end{fulllineitems}



\begin{fulllineitems}
\pysiglinewithargsret{\sphinxbfcode{\sphinxupquote{points}}}{\emph{word}}{}
Calculates points in a given word.
\begin{quote}
\begin{description}
\item[{Args:}] \leavevmode
word (string): User answer.

\item[{Returns:}] \leavevmode
points (int): To be added to total score. Based on length of string.

\end{description}
\end{quote}

\end{fulllineitems}



\begin{fulllineitems}
\pysiglinewithargsret{\sphinxbfcode{\sphinxupquote{solve\_board}}}{}{}
Recursive function that traverses current boggle board and
updates game\_solutions with possible words.

\end{fulllineitems}



\begin{fulllineitems}
\pysiglinewithargsret{\sphinxbfcode{\sphinxupquote{verify}}}{\emph{word}}{}
Determines if word is a valid answer.
\begin{quote}\begin{description}
\item[{Parameters}] \leavevmode
\sphinxstyleliteralstrong{\sphinxupquote{word}} (\sphinxstyleliteralemphasis{\sphinxupquote{string}}) \textendash{} User answer.

\item[{Returns}] \leavevmode
bool

\end{description}\end{quote}

\end{fulllineitems}



\begin{fulllineitems}
\pysiglinewithargsret{\sphinxbfcode{\sphinxupquote{words\_from\_start}}}{\emph{board}, \emph{i}, \emph{j}, \emph{trie}}{}
Returns set of possible words starting from position (i,j) in the board.
Uses Trie to verify word.
\begin{quote}\begin{description}
\item[{Parameters}] \leavevmode\begin{itemize}
\item {} 
\sphinxstyleliteralstrong{\sphinxupquote{board}} (\sphinxstyleliteralemphasis{\sphinxupquote{list}}) \textendash{} A two-dimensional array holding the NxN game board.

\item {} 
\sphinxstyleliteralstrong{\sphinxupquote{i}} (\sphinxstyleliteralemphasis{\sphinxupquote{int}}) \textendash{} The row index of beginning letter.

\item {} 
\sphinxstyleliteralstrong{\sphinxupquote{j}} (\sphinxstyleliteralemphasis{\sphinxupquote{int}}) \textendash{} The column index of the beginning letter.

\item {} 
\sphinxstyleliteralstrong{\sphinxupquote{trie}} (\sphinxstyleliteralemphasis{\sphinxupquote{Trie}}) \textendash{} Dictionary Trie of the possible words given dictionary file

\end{itemize}

\item[{Returns}] \leavevmode
Possible words starting from (i,j).

\item[{Return type}] \leavevmode
solutions (set)

\end{description}\end{quote}

\end{fulllineitems}


\end{fulllineitems}



\section{modules/trie.py}
\label{\detokenize{index:modules-trie-py}}

\begin{fulllineitems}
\pysiglinewithargsret{\sphinxbfcode{\sphinxupquote{class }}\sphinxcode{\sphinxupquote{trie.}}\sphinxbfcode{\sphinxupquote{Trie}}}{\emph{listy=None}}{}
Data structure that allows for faster verification of word combinations through
a collection of TrieNodes.
\begin{quote}\begin{description}
\item[{Variables}] \leavevmode\begin{itemize}
\item {} 
\sphinxstyleliteralstrong{\sphinxupquote{root}} (\sphinxstyleliteralemphasis{\sphinxupquote{TrieNode}}) \textendash{} Root of current trie.

\item {} 
\sphinxstyleliteralstrong{\sphinxupquote{listy}} (\sphinxstyleliteralemphasis{\sphinxupquote{list}}) \textendash{} Dictionary of words. Read from .txt file in Engine class

\end{itemize}

\end{description}\end{quote}


\begin{fulllineitems}
\pysiglinewithargsret{\sphinxbfcode{\sphinxupquote{add\_to\_trie}}}{\emph{listy}}{}
Adds given list of words to trie
\begin{quote}\begin{description}
\item[{Parameters}] \leavevmode
\sphinxstyleliteralstrong{\sphinxupquote{listy}} (\sphinxstyleliteralemphasis{\sphinxupquote{list}}) \textendash{} Refer to Trie class.

\end{description}\end{quote}

\end{fulllineitems}


\end{fulllineitems}



\begin{fulllineitems}
\pysiglinewithargsret{\sphinxbfcode{\sphinxupquote{class }}\sphinxcode{\sphinxupquote{trie.}}\sphinxbfcode{\sphinxupquote{TrieNode}}}{\emph{value}}{}
Represents node in Trie class.
\begin{quote}\begin{description}
\item[{Variables}] \leavevmode\begin{itemize}
\item {} 
\sphinxstyleliteralstrong{\sphinxupquote{value}} (\sphinxstyleliteralemphasis{\sphinxupquote{string}}) \textendash{} Letter contained in node.

\item {} 
\sphinxstyleliteralstrong{\sphinxupquote{children}} (\sphinxstyleliteralemphasis{\sphinxupquote{list}}) \textendash{} List of TrieNodes that are children of current TrieNode.

\item {} 
\sphinxstyleliteralstrong{\sphinxupquote{complete}} (\sphinxstyleliteralemphasis{\sphinxupquote{string}}) \textendash{} ‘None’ if TrieNode does not complete a word.
Otherwise, contains completed word, which is the concatenation of all past

\item {} 
\sphinxstyleliteralstrong{\sphinxupquote{values.}} (\sphinxstyleliteralemphasis{\sphinxupquote{TrieNode}}) \textendash{} 

\end{itemize}

\end{description}\end{quote}


\begin{fulllineitems}
\pysiglinewithargsret{\sphinxbfcode{\sphinxupquote{add}}}{\emph{child}}{}
Add child to the node

\end{fulllineitems}



\begin{fulllineitems}
\pysiglinewithargsret{\sphinxbfcode{\sphinxupquote{get\_child}}}{\emph{value}}{}
Find direct child of current node with given value
\begin{quote}\begin{description}
\item[{Parameters}] \leavevmode
\sphinxstyleliteralstrong{\sphinxupquote{value}} (\sphinxstyleliteralemphasis{\sphinxupquote{string}}) \textendash{} Refer to TrieNode.

\item[{Returns}] \leavevmode
If requested value is present in direct children.
None: Value is not present

\item[{Return type}] \leavevmode
child (TrieNode)

\end{description}\end{quote}

\end{fulllineitems}


\end{fulllineitems}



\section{modules/ascii\_art.py}
\label{\detokenize{index:modules-ascii-art-py}}

\begin{fulllineitems}
\pysigline{\sphinxbfcode{\sphinxupquote{class }}\sphinxcode{\sphinxupquote{ascii\_art.}}\sphinxbfcode{\sphinxupquote{Ascii}}}
Contains ASCII art to be used in Interface. Attributes title,
success, and title data sourced from: \sphinxurl{http://patorjk.com/software/taag/}.
\begin{description}
\item[{Attribues:}] \leavevmode
title (list): Title shown in start and difficulty menu.
success (list): Text indicator if word is valid.
error (list): Text indicator if word is invalid.
binary\_sm (list): Decorative text.
jumbled\_char (list): Decorative text.

\end{description}

\end{fulllineitems}

\chapter{Finding Solutions Given Board}

\section{Source} 
\begin{fulllineitems}
Game logic was based heavily off: https://github.com/macallmcqueen/boggle-solver-using-trie.git
\end{fulllineitems}

\section{Short Explanation}
\begin{fulllineitems}
Engine creates a \href{https://en.wikipedia.org/wiki/Trie}{Trie} data structure, which stores the possible words efficiently not only in terms of memory but look-up as well. \\ \\
Board is traversed one-by-one from (0,0) and recursively looks for possible paths (which make up a string) by "going-to" adjacent nodes. Each path is then checked if it completes a word via the Trie class. Path is ignored (program moves on to the next possible path) if it reaches a dead-end (if next letter makes string invalid according to Trie).
\end{fulllineitems}
\begin{fulllineitems}
\item{Further Reading:}\\
\href{https://www.geeksforgeeks.org/trie-insert-and-search/}{Geeks for Geeks}\\
\href{https://medium.com/basecs/trying-to-understand-tries-3ec6bede0014}{Medium Article}
\end{fulllineitems}

\renewcommand{\indexname}{Index}
\printindex
\end{document}